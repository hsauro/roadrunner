\documentclass[12pt, letter, oneside]{book}

\usepackage{fancyhdr, ifpdf}
\pagestyle{fancyplain}

%% some redefination of the headers and footers
\renewcommand{\chaptermark}[1]%
                 {\markboth{#1}{}}
\renewcommand{\sectionmark}[1]%
                 {\markright{\thesection\ #1}}
\lhead[\fancyplain{}{\thepage}]%
      {\fancyplain{}{\rightmark}}
\rhead[\fancyplain{}{\leftmark}]%
      {\fancyplain{}{\thepage}}
\cfoot{}
\sloppy

\ifpdf
   \usepackage[pdftex]{graphicx}
   \pdfinfo {
      /Title (RoadRunner Introduction)
      /Subject (Documentation)
      /Author (Totte Karlsosn)
      /Keywords Synthetic Biology
   }
\else
   \usepackage{graphicx}
\fi



%% start the show...
\begin{document}

\author{Totte Karlsson and Herbert Sauro
\\\\University of Washington, Seattle Washington USA}
\title{Introduction to Simulations of Chemical and Biological Systems 
\\ RoadRunner Version 1.0
}
\maketitle

\pagenumbering{roman}

\setcounter{page}{1}
\tableofcontents
%\listoffigures
%\listoftables
\pagenumbering{arabic}
\chapter{Introduction}
\thispagestyle{empty}

purpose\\
history\\
present\\
future\\




\chapter{The Software}
\emph{A simulation environment for computer models \\
	representing chemical and biological processes}

\section{Core Architecture}
The Roadrunner architecture aims at providing the end user with a concise and well defined API for solving problems in the semantic domain of Systems Biology and Chemistry. \\
In a few words, Roadrunner reads a theoretical model, with parameters on input, propagates the model in time, producing data simulating how the system evolves with time (Fig 1).
\\
RoadRunner do provide the user with many specific functions returning specifics of a currently loaded model, such as number of species, number of reactions etc...
   
\subsection{Building blocks}
RoadRunner's core is written in C++, and uses various object oriented concepts such as classes, inheritance and runtime type information for its execution. The software core do consist of about 100 C++ classes. Only a few, the important one will be discussed here. \\
The main class is called RoadRunner and is found in the the C++ header file named rrRoadRunner.h.\\ RoadRunner is a monolithic class that is the entry point for almost all RoadRunner activity and functionality. As such, it has many data members, that are themselves substantial classes, e.g. CVOdeInterface, ModelGenerator, NOMSupport, PluginManager to name a few.
  
\\
\subsection{Dependencies}
\subsubsection{Data flow}
\subsubsection{Functional Components}
\subsubsection{Base}
\subsubsection{Threading}
\subsubsection{plugins}
\subsubsection{Diagrams}
Model Generation
Model Creation
Model Simulation
\section{Wrappers}
\subsection{Introduction}
\subsection{Handles}
\subsection{Consistent api}
\subsection{C and Python}
\subsection{Handles}
\subsection{RRHandle}
\subsection{Data Handle}
\subsection{Examples}
\subsection{Plugin Framework }
\section{Introduction}
\subsection{Examples}
\subsection{Plugin API}
\subsubsection{Create your own plugin}
\subsubsection{C++}
\subsubsection{C}
\subsubsection{Python}
\subsubsection{Delphi}
\subsection{Examples}
\subsection{Intro}
\subsubsection{Using Core functionalities}
\subsubsection{Plugins}
\subsubsection{Distributed plugins }
\subsubsection{Minimization Plugins}
\subsubsection{Levenberg-Marquardt}
\section{Appendix}
\subsection{Getting the source}
\subsection{Building}
\subsection{3rd party}
\subsection{core}
\subsection{plugins}
\subsection{wrappers}
\subsubsection{C}
\subsubsection{Python}
\subsection{Testing}
\subsection{Intro}
\subsubsection{C api}
\subsubsection{Python}
\subsection{Other sources of information}
\subsubsection{Google Code}
\subsubsection{FlySpray}
\subsubsection{sysbio home page}
\subsubsection{My blog}
\subsubsection{Summary}
\subsubsection{Examples}

\begin{appendix}
\appendix
\chapter{Some Appendix}
\label{someapp}

nothing special to say here...


\end{appendix}

\addcontentsline{toc}{chapter}{Bibliography}

\bibliographystyle{plain}
\thispagestyle{empty}
\bibliography{demo}

\end{document}






